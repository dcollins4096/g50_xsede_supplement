
\def\request{4352 SU}

We are requesting \request\ to supplement our grant.  This supplement will allow
us to continue progressing while we await the next proposal review.

During the work on the \nameTurbulence\ project, it was discovered that these
results extend nicely to magnetized studies.  We have begun four MHD simulations,
and will continue these during the period between August and October.  

The four simulations are driven turbulence with uniform mean magnetic fields.
In these simulations, kinetic energy is added to the simulation at a controlled
rate.  This energy is added to the large scale of the simulation, and the
nonlinear dynamics of the system cause the energy to cascade to smaller scales.
Energy is then dissipated by the numerics at the smallest scale.    The
important information carrying regime is the \emph{inertial range}, between the
driving and dissipation scales.  The simulations all have an r.m.s. velocity of 5 $c_s$,
where $c_s$ is the speed of sound in the cloud, magnetic fields such that the
ratio of the r.m.s. velocity to magnetic velocity is 1/2, 1, 2, and 3.  Thus the
simulations range from strongly magnetized to weakly.

The resolution of these simulations is $512^3$.  This resolution represents a
balance between cost and benefit; large enough to resolve a inertial range, but
small enough to be run relatively quickly.  

The simulation time for the simulations is based on the pattern turn-over time, the time
for the large-scale velocity pattern to replace itself.  We define the dynamical
time as
$t_{dyn} =L/V$, where $L$ is the pattern size and $V$ is the r.m.s velocity in
the box.  During a simulation, it takes roughly two $t_{dyn}$ for the energy to
fully cascade to small scales.  We then run for another three $t_{dyn}$ to
establish statistics.  Turbulence is a chaotic process, and any given snapshot
is not necessarily representative of the trends.  Averaging over three dynamical
times allows us to have converged statistics.

The four simulations are being run on \emph{Stampede 2}.  They are each using 8
nodes, with 64 cores per node.  They have each had one day of simulations and
are at $t=1 t_{dyn}$, so will need four more restarts each.  A more elaborate
estimate of the timing can be found in the original proposal.  Thus the total
supplement request is 
\begin{align}
8 \rm{nodes} \times 4 \rm{sims} \times 4 \rm{restarts} \times 24 \rm{hrs} =
3072 SUs.
\end{align}

In addition to simulation time, we are requesting an additional 1440 SU for
analysis.  Analysis is done both with interactive and batch sessions, and is
done one node at a time.  There are roughly 40 business days in August and
September, the period the supplement will cover, and if the three grad students
and one PI are working for 8 hours per day, that yields 
\begin{align}
    40 \rm{days}\ \times 4 \rm{people} \times 8 \rm{hours}=1280 \rm{SUs}
\end{align}
hours for analysis time.
