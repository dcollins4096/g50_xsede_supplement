%\documentclass[11pt]{article}  % for e-submission to ApJ
\documentclass[11pt]{NSF}  % for e-submission to ApJ

%\documentclass[12pt,preprint2]{aastex}  % for e-submission to ApJ - two column

%\documentclass[manuscript]{emulateapj}  % this makes everything look like ApJ

\usepackage{graphicx, natbib, color, bm, amsmath, epsfig}
\input{aas_papers.tex}
\input{commands.tex}

\def\tfinal{T_{\rm{final}}}
\def\suzu{\ensuremath{\rm{SU}_{\rm{zone}}}}
\def\nzones{N_{\rm{z},\ell}}
\newcommand{\Lund}{\ensuremath{\mathrm{S}}}
%\newcommand{\SUestimate}[1]{\textcolor{red}{#1}}
\newcommand{\SUestimate}[1]{#1}
\def\SUtotal{5\sci{4}}
\def\SUsn{1.9\sci{4}}
\def\SUcore{1.6\sci{4}}
\def\SUturb{1.5\sci{4}}
\def\DiskCore{1.5\sci{4} Gb}
\def\DiskSNe{1.7\sci{4}Gb}
\def\DiskTurb{3.2\sci{3} Gb}
\def\StoreTotal{3.5\sci{4} Gb}
\def\nameCores{\emph{cores}}
\def\nameSupernova{\emph{supernovae}}
\def\nameTurbulence{\emph{turbulence}}
\def\suPerZoneUpCores{\ensuremath{6.2\sci{-11}}}
\def\suPerZoneUpSupernova{\ensuremath{5.2\sci{-11}}}
\def\suPerZoneUpTurbulence{\ensuremath{4.4\sci{-11}}}

\citestyle{aa}  % correct formatting for ApJ style files

\usepackage{aas_macros}
\begin{document}

\begin{centering}
\begin{LARGE}
Progress Report for

TG-AST140008

\end{LARGE}
\end{centering}


\pagestyle{plain}

During our 2021-2022 allocation we have primarily worked on analysis of results and writing papers.
Four papers
from the \nameCores\ project are in preparation.
One paper on \nameSupernova\ has been published as \citep{Hristov21}.
Three papers from the
\nameTurbulence\ project have been produced, one of those has been submitted to Monthly
Notices of the Royal Astronomical Society.

The first paper in the \nameCores\ project examines the initial conditions 
of star forming clouds.  We find an abundance of fractal structures, mixing
between different cores early on, and we find a novel prediction of the star
formation rate.  This will be submitted by August 2022 as Collins, Le, and
Jimenez (2022).

The second paper in the \nameCores\ project examines the rate of collapse and
gravitational binding energy during collapse.
  We find that the cores do not always have a
phase where the velocity is subsonic, as was originally expected; and the
collapse is slower than free-fall in all but the largest objects, which are
substantially faster.  This will be submitted as Le, Collins and Jimenez (2022).

The third paper, to be submitted by Summer 2022,
examines the behavior of magnetic fields during the collapse.   It is found that
the ratio of magnetic field strength to density decreases as a function of time
by the act of turbulence alone.  This will be submitted as Jimenez, Collins and
Le (2022).

The fourth publication, to be completed by Fall 2022, is an examination of the
full suite of forces acting on collapsing gas.  

The \nameTurbulence\ study has netted one submitted publication and two more
being finalized.  The first is a study of the PDF of density in isothermal
turbulence, and has been submitted to Montly Noticess of the Royal Astronomical
Society as Rabatin and Collins (2022).  The second is
a study of the joint distribution of the density and velocity and their
correlation.  The thrid is the preliminary suite that inspires the current
proposal, a prediction of the joint distribution of Kinetic and Internal energy
in isothermal turbulence.  


We concluded the Supernova project this year, with the results published
in \citep{Hristov21}.  We find the surprising result that many Type Ia spernovae
require very large ($10^6$G), magnetic fields to reproduce the late-time light
curves observed.  

\bibliographystyle{apj}
\bibliography{apj-jour,ms.bib}  % looks in ms.bib for bibliography info

\end{document}  


