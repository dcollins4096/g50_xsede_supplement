\section{Timing}
\begin{table} \begin{center} \caption{The accounting for the request. The total cost for each simulation,
$SU=\frac{N_Z N_U N_N}{N_C \zeta}\frac{1}{3600}$.  For the proposed 12
simulations, $N_Z=1024^3$, $N_N=64$, $N_C=4096$, $\zeta=10^5=$(core
updates)/(processor second).  $N_U=T/\Delta t$ is the number of steps, and
depends on \mach\ as described in the text.  $T_{wall}=SU/N_N$ is presented in hours.}
 \label{table2}                                                                                                                                               
\begin{tabular}{               c               c               r                       r                       r               r               r       }       
                 $\Mach$       &   $\xi$       &       T               &$\Delta T$               &     \Nu       &$T_{wall}$       &      SU             \\
  \hline                                                                                                                                               
                     1.0       &     0.0       &       5               &1.4\sci{-5}               &3.7\sci{5}       &   269.7       &1.7\sci{4}             \\
                     2.0       &     0.0       &     2.5               &9.0\sci{-6}               &2.8\sci{5}       &   202.3       &1.3\sci{4}             \\
                     4.0       &     0.0       &    1.25               &5.4\sci{-6}               &2.3\sci{5}       &   168.6       &1.1\sci{4}             \\
                    10.0       &     0.0       &     0.5               &2.5\sci{-6}               &2.0\sci{5}       &   148.3       &9.5\sci{3}             \\
                     1.0       &     0.5       &       5               &1.4\sci{-5}               &3.7\sci{5}       &   269.7       &1.7\sci{4}             \\
                     2.0       &     0.5       &     2.5               &9.0\sci{-6}               &2.8\sci{5}       &   202.3       &1.3\sci{4}             \\
                     4.0       &     0.5       &    1.25               &5.4\sci{-6}               &2.3\sci{5}       &   168.6       &1.1\sci{4}             \\
                    10.0       &     0.5       &     0.5               &2.5\sci{-6}               &2.0\sci{5}       &   148.3       &9.5\sci{3}             \\
                     1.0       &     1.0       &       5               &1.4\sci{-5}               &3.7\sci{5}       &   269.7       &1.7\sci{4}             \\
                     2.0       &     1.0       &     2.5               &9.0\sci{-6}               &2.8\sci{5}       &   202.3       &1.3\sci{4}             \\
                     4.0       &     1.0       &    1.25               &5.4\sci{-6}               &2.3\sci{5}       &   168.6       &1.1\sci{4}             \\
                    10.0       &     1.0       &     0.5               &2.5\sci{-6}               &2.0\sci{5}       &   148.3       &9.5\sci{3}             \\
  \hline                                                                                                                                               
               &                               &                       &                       &               &Total SU       &1.5\sci{5}             \\
               &                               &                       &                       &               &Disk (Gb)       &7.7\sci{4}             \\
\end{tabular}                                                                                                                                               
\end{center}                                                                                                                                               
\end{table}                                                                                                                                                

The total cost can be found as
\begin{align}
SU &= T_{wall} N_N\\
T_{wall} &= \frac{N_Z N_U}{N_C \zeta} \frac{1}{3600},
\end{align}
where $N_N$ is the number of nodes, $N_Z$ is the number of zones, $N_U$ is the number of updates, $N_C$ the
number of cores, and $\zeta$ = (zone-updates)/(core-second) is the performance of
the code.  $N_Z$ is set by our target resolution of $1024^3$.  From the scaling
study, we find that $\zeta=10^5$.  Due to the excellent scaling of
fixed-resolution Enzo, we will use $N_C=4096$.
We will use 64 cores per node and 4096
cores, so $N_N=64$.
$N_U$ is the number of updates,
which depends on \Mach\ in the following way.

The number of updates is found by $N_U=T_{sim}/\Delta t$.  $T_{sim}=10\tdyn$ per
our noise requirement.  The time step size, $\Delta t$, is determined by a
typical Courant condition that the signal cannot propagate more than half a zone
in a timestep,
\begin{align}
\Delta t &= \eta \frac{\Delta x}{v_{max}+ c_s}\\
&\propto \frac{1}{1+\Mach}
\end{align}
where $\Delta x$ is the zone size, and $v_{max}+c_s$ is the maximum signal speed over
the whole domain.
It was verified with our suite of Mach 8 $1024^3$ runs that $\Delta t$ does not
depend on the forcing parameter, $\xi$.  
The peak velocity, $v_{max}$, is not predictable due to the chaos of the
turbulence, but is thankfully found to scale with the Mach number, and we
calibrate to our recent high res simulations.


